% Package comment legend:
% !! = package loading order
% ++ = instructions to avoid conflicts
% CC = package conflicts. Sometimes can be avoided by renaming commands
% RR = package overrides other package
% DD = package dependency, loads other package if not already loaded
% dd"" = nested package notes. "" applies to loaded package
% SS = package supersedes feature in other package
% ** = important note when using
% <> = might need to move package around
% Todo = have yet to research package technicalities

% Note: it may be better to explicitly put the loaded packages in the \usepackage commands like \usepackage{amsmath, mathtools} = \usepackage{mathtools} to make things clearer
% Note: these sections of packages may have to be rearranged...they might not even stay contiguous
% Note: http://www.macfreek.nl/memory/LaTeX_package_conflicts has an incomplete listing of some conflicts.

\RequirePackage[l2tabu, orthodox]{nag} % tells you about incorrect and technically legal but probably unwanted latex
\RequirePackage{etex} % avoid warnings. etex is not needed in modern latex


\documentclass[letterpaper, 10pt]{article} % default [10pt] can be changed

% ------Colors------------------------
% Must put before graphics

\usepackage[hyperref, dvipsnames*, svgnames, x11names]{xcolor} % ** if using pdftex, apply option fixpdftex
% ** dvipsnames conflicts with some other colors; thus, we defer their definition

% ------------------------------------

% ------Tools--------------
\usepackage[xetex]{graphicx} % for including graphics in pdfTeX
% !! load before xltxtra

\usepackage{verbatim} % LaTeX required, extends verbatim and adds comment environment
% almost certainly does not conflict with anything

% \usepackage{afterpage} % a LaTeX required package. Allows execution of commands after the next page break
% this should not conflict

\usepackage{calc} % this package is a LaTeX required package, shouldn't conflict

\usepackage{etoolbox}
% \usepackage{etextools}
% DD loads etoolbox
% CC conflicts with etoolbox, it redefines \doforlistloop
\usepackage{xpatch} % DD loads on etoolbox
% extended macro-patching commands

% -------------------------


% ---Default ams packages (big happy family)-----------

\usepackage{mathtools} %DD loads amsmath
% I would not remove this because it fixes 2 bugs in amsmath automatically.
% The most important part of mathtools is \clap, \mathclap for smooshing both sides of a box without affecting the ink, which is useful for long subscripts and superscripts.
% !! must come before unicode-math since unicode-math overwrites
\mathtoolsset{mathic=true}

% \usepackage{amssymb} % not needed with unicode-math

% \usepackage{colonequals} % more colon symbols
% !! must come before unicode-math since unicode-math overwrites
% -----------------------------------------------------

% ----Encoding (not xetex or luatex) --------------------
% Doing XeTeX or LuaTeX pretty much obviates the need for these
% \usepackage[T1]{fontenc} % even if you're not writng other languages, if you're using LaTeX, you need this. Even < and > display wrongly!

% \usepackage[utf8]{inputenc} %** Note that just because you can input unicode doesn't mean LaTeX will recognize the characters.
% <> (maybe) It is remotely possible that defining new fonts before this package lets you use more symbols.

% ---- (xetex or luatex) --------------------------------

\usepackage{xltxtra} %** fontspec necessary for xetex/luatex
% DD pulls in fontspec, metalogo, realscripts
\usepackage{unicode-math} % unicode input

% --------------------------------

% -----Theorems-----

\usepackage[ntheorem]{empheq} % side effect: fixes ntheorem, improves endmarks
% \usepackage{amsthm} % CC conflicts with ntheorem
\usepackage[noconfig, thmmarks, amsmath, hyperref]{ntheorem} %CC conflicts with amsthm. Use option amsthm instead for amsthm-like commands.
% Todo may be buggy. It has more features than amsthm though
% option standard conflicts with option amsthm

% ------------------

% ----General features--------------
% \usepackage{siunitx} % Takes care of all your units-related needs.

\usepackage{array} % a LaTeX required package, shouldn't conflict. Extends array and tabular environments
\usepackage{delarray} % a LaTeX required package. Extends array, allowing it to be surrounded by delimiters. Definitely safe to use with array

\usepackage{parskip}
% instead of indenting paragraphs, puts 0.5\baselineskip between them
% I think LaTeX's default paragraphs are fine, but others may disagree

\usepackage[shortlabels]{enumitem} % allows customization of list environments. Essential.
% -------------------------------

% -----Graphics-------
% These shouldn't conflict with any other section, as far as I can tell.

% \usepackage{tikz} % tikz/pgf! Very rewarding.
% DD loads xcolor, better give your colors to xcolor before that happens
% ** You should probably use the built-in library babel that sets the catcodes of various symbols correctly in all tikzpicture enviroments.
% \usetikzlibrary{babel, calc} % babel necessary to protect catcodes, especially if using calc

% --------------------


% ----Typography------------------

\usepackage{microtype} % extremely, extremely subtle changes to make text look better. I haven't researched conflicts yet.

% --------------------------------

% ---Text formatting--------
% \usepackage{soul} % spaced out letters, underlining, strikethrough, highlighting
% I haven't found any conflicts yet
% Quality of package is dubious

% \usepackage{framed} % need to look into this one more

% --------------------------

% ---Math mode features-------------

\usepackage{mleftright} % defines \mleft, \mright to be delimiters, not an inner atom. This fixes spacing issues when any operator or function is placed in front of \left
\mleftright% changes \left to \mleft and \right to \mright

\usepackage{centernot} % puts \not centered on the previous symbol

% \usepackage{scalerel} % provides commands to scale or stretch objects either relative to another object or absolutely

% \usepackage{accents} % note that accents redefines dddot and ddddot
% \newcommand{\vect}[1]{\accentset{\rightharpoonup}{#1}} % I have learned that there are better ways to do this. See the (lengthy) discussion of how to put things on top of other things on tex.stackexchange.com . You can start with one LaTeXer's implementation of "widebar" at http://tex.stackexchange.com/questions/16337/can-i-get-a-widebar-without-using-the-mathabx-package/60253#60253 . It's simply ingenious.

% \usepackage{mathdots} %RRCC overrides, possibly conflicts with accents because it redefines dddot and ddddot (again)

% \neverevereverusepackage{commath} % Please never, ever, ever use this package. Its formatting is literally wrong, in several ways.

\usepackage[italic, thinc]{esdiff} % \diff for derivatives

% ------------------------------------
% --References--------------

\usepackage[hypertexnames=false]{hyperref} %!! One of the most problematic packages with respect to loading order. Sometimes needs to be loaded before, sometimes after. See the readme at http://www.ctan.org/pub/tex-archive/macros/latex/contrib/hyperref/README for a comprehensive listing and see http://www.macfreek.nl/memory/LaTeX_package_conflicts for more conflicts.
% DD loads backref, nameref

\usepackage{cleveref} %!! load very late. Load all other reference packages before this
% can use \cref{} which adds text depending on what is referenced.
% SS Compatible with fancyref, ntheorem, varioref, but supersedes many features of them

\usepackage{autonum} %!! must be loaded after other reference packages and cleveref. load very late
% ++ must load hyperref with option hypertexnames=false
% only number the referenced equations; compatible with cleverref. It's actually a very new package (it came out Jan 2015).
% SS Supersedes the same feature in mathtools.
% ** requires 3 compilations total (1 extra) to determine references
% ** provide \label at end of equation, before \\, except at the start for split
% DD requires package etextools. According to developer, this package is experimental
% ddCC etextools conflicts with etoolbox
% ddCC etextools conflicts with biblatex. Consult http://tex.stackexchange.com/questions/220268/biblatex-and-autonum-dont-work-together and http://tex.stackexchange.com/questions/49556/incompatibility-between-etextools-and-etoolbox-command-dolistloop-forlistloop

% --------------------------

% ----Page layout-------------------
\usepackage[xetex,margin=1in]{geometry} % can change the margins, etc. of the page
% !! in some cases, must come after hyperref

% ----------------------------------

% --------------------------Macros----------

% \tracingpatches % traces patches

%https://tex.stackexchange.com/questions/19457/using-display-style-fraction-in-a-matrix-environment
\makeatletter
\newif\ifcenter@asb@\center@asb@false
\def\center@arstrutbox{%
  \setbox\@arstrutbox\hbox{$\vcenter{\box\@arstrutbox}$}%
}
\newcommand*{\CenteredArraystretch}[1]{%
  \ifcenter@asb@\else
  \patchcmd{\@array}{\begingroup \@mkpream}{\center@arstrutbox \begingroup \@mkpream}{}{}%
  \center@asb@true
  \fi
  \renewcommand{\arraystretch}{#1}%
}
\makeatother

% https://tex.stackexchange.com/questions/16337/can-i-get-a-widebar-without-using-the-mathabx-package
\makeatletter
\let\save@mathaccent\mathaccent
\newcommand*\if@single[3]{%
  \setbox0\hbox{${\mathaccent"0362{#1}}^H$}%
  \setbox2\hbox{${\mathaccent"0362{\kern0pt#1}}^H$}%
  \ifdim\ht0=\ht2 #3\else #2\fi
}
% The bar will be moved to the right by a half of \macc@kerna, which is computed by amsmath:
\newcommand*\rel@kern[1]{\kern#1\dimexpr\macc@kerna}
% If there's a superscript following the bar, then no negative kern may follow the bar;
% an additional {} makes sure that the superscript is high enough in this case:
\newcommand*\widebar[1]{\@ifnextchar^{{\wide@bar{#1}{0}}}{\wide@bar{#1}{1}}}
% Use a separate algorithm for single symbols:
\newcommand*\wide@bar[2]{\if@single{#1}{\wide@bar@{#1}{#2}{1}}{\wide@bar@{#1}{#2}{2}}}
\newcommand*\wide@bar@[3]{%
  \begingroup
  \def\mathaccent##1##2{%
    % Enable nesting of accents:
    \let\mathaccent\save@mathaccent
    % If there's more than a single symbol, use the first character instead (see below):
    \if#32 \let\macc@nucleus\first@char \fi
    % Determine the italic correction:
    \setbox\z@\hbox{$\macc@style{\macc@nucleus}_{}$}%
    \setbox\tw@\hbox{$\macc@style{\macc@nucleus}{}_{}$}%
    \dimen@\wd\tw@
    \advance\dimen@-\wd\z@
    % Now \dimen@ is the italic correction of the symbol.
    \divide\dimen@ 3
    \@tempdima\wd\tw@
    \advance\@tempdima-\scriptspace
    % Now \@tempdima is the width of the symbol.
    \divide\@tempdima 10
    \advance\dimen@-\@tempdima
    % Now \dimen@ = (italic correction / 3) - (Breite / 10)
    \ifdim\dimen@>\z@ \dimen@0pt\fi
    % The bar will be shortened in the case \dimen@<0 !
    \rel@kern{0.6}\kern-\dimen@
    \if#31
    \overline{\rel@kern{-0.6}\kern\dimen@\macc@nucleus\rel@kern{0.4}\kern\dimen@}%
    \advance\dimen@0.4\dimexpr\macc@kerna
    % Place the combined final kern (-\dimen@) if it is >0 or if a superscript follows:
    \let\final@kern#2%
    \ifdim\dimen@<\z@ \let\final@kern1\fi
    \if\final@kern1 \kern-\dimen@\fi
    \else
    \overline{\rel@kern{-0.6}\kern\dimen@#1}%
    \fi
  }%
  \macc@depth\@ne
  \let\math@bgroup\@empty \let\math@egroup\macc@set@skewchar
  \mathsurround\z@ \frozen@everymath{\mathgroup\macc@group\relax}%
  \macc@set@skewchar\relax
  \let\mathaccentV\macc@nested@a
  % The following initialises \macc@kerna and calls \mathaccent:
  \if#31
  \macc@nested@a\relax111{#1}%
  \else
  % If the argument consists of more than one symbol, and if the first token is
  % a letter, use that letter for the computations:
  \def\gobble@till@marker##1\endmarker{}%
  \futurelet\first@char\gobble@till@marker#1\endmarker
  \ifcat\noexpand\first@char A\else
  \def\first@char{}%
  \fi
  \macc@nested@a\relax111{\first@char}%
  \fi
  \endgroup
}
\makeatother

% --------------------------

% --Settings--

\allowdisplaybreaks

% ------------

% --Macros--

\DeclarePairedDelimiter{\abs}{\lvert}{\rvert} % this command defined in mathtools
\DeclarePairedDelimiter{\norm}{\lVert}{\rVert}
\DeclarePairedDelimiter{\parens}{(}{)}
\DeclarePairedDelimiter{\bracks}{[}{]}
\DeclarePairedDelimiter{\braces}{\{}{\}}

\DeclarePairedDelimiter{\ceil}{\lceil}{\rceil}
\DeclarePairedDelimiter{\floor}{\lfloor}{\rfloor}

\newcommand{\naturals}{\mathbb{N}}
\newcommand{\integers}{\mathbb{Z}}
\newcommand{\rationals}{\mathbb{Q}}
\newcommand{\reals}{\mathbb{R}}
\newcommand{\complexes}{\mathbb{C}}

\newcommand{\pnaturals}{\naturals_{>0}}
\newcommand{\nnnaturals}{\naturals_0}
\newcommand{\pintegers}{\integers_{>0}}
\newcommand{\nnintegers}{\integers_{≥0}}
\newcommand{\preals}{\reals_{>0}}
\newcommand{\nnreals}{\reals_{≥0}}

\newcommand{\percent}{\mathbin{\%}}

\let\ReFrak\Re % save old definitions
\let\ImFrak\Im
\renewcommand{\Re}{\operatorname{Re}}
\renewcommand{\Im}{\operatorname{Im}}
\DeclareMathOperator{\im}{im}
\DeclareMathOperator{\tr}{tr}
\DeclareMathOperator{\Conv}{Conv}
\DeclareMathOperator{\absop}{abs}

\newcommand{\phantomrel}{\mathrel{\hphantom{=}}{}}

\renewcommand{\Pr}{\mathbb{P}}
\newcommand{\E}{\mathbb{E}}

\newcommand{\ud}{\relax}
\let\ud\d % save old d, which is an underdot accent
% simplification of dx notation
\renewcommand{\d}[1]{\mathinner{d#1}}

% ----------

% ----Theorem stuff----

\theoremclass{LaTeX}
\theoremstyle{plain}
\theoremheaderfont{\normalfont\bfseries}
\theorembodyfont{\normalfont}
\theoremseparator{.}
\theorempreskip{0.75\baselineskip}%\topsep
\theorempostskip{0.75\baselineskip}%\topsep
\theoremindent0cm
\theoremnumbering{arabic}
\theoremsymbol{}
\newtheorem{thm}{Theorem}

\theoremclass{thm}
\newtheorem{lem}[thm]{Lemma}

\theoremclass{thm}
\newtheorem{prop}[thm]{Proposition}

\theoremclass{thm}
\newtheorem{cor}{Corollary}[thm]

\theoremclass{thm}
\theoremprework{\bigskip\hrule}
\theorempostwork{\hrule\bigskip}
\newtheorem{prob}{Problem}

\theoremclass{thm}
\theoremheaderfont{\normalfont\itshape}
\theoremstyle{nonumberplain}
\theoremsymbol{\ensuremath{■}}
\theorempostwork{\setcounter{case}{0}}
\newtheorem{prf}{Proof}

\theoremclass{prf}
\theoremheaderfont{\normalfont\bfseries}
\theoremstyle{nonumberbreak}
\theoremsymbol{\ensuremath{✓}}
\theorempostwork{\setcounter{case}{0}}
\newtheorem{sol}{Solution}

\theoremclass{prf}
\theoremstyle{plain}
\theoremindent\leftmargin
\theoremsymbol{\ensuremath{▶}}
\theorempostwork{\setcounter{case}{0}}
\newtheorem{clm}{Claim}[prf]

\theoremclass{clm}
\theorempostwork{\setcounter{case}{0}}
\newtheorem{solclm}{Claim}[sol]

\theoremclass{clm}
\theoremsymbol{\ensuremath{▷}}
\newtheorem{case}{Case}

\theoremclass{clm}
\theoremnumbering{alph}
\theoremsymbol{\ensuremath{▷▷}}
\newtheorem{subcase}{Subcase}[case]

\theoremclass{clm}
\theoremstyle{nonumberplain}
\theoremsymbol{}
\newtheorem{basecase}{Base Case}

\theoremclass{basecase}
\newtheorem{indstep}{Inductive Step}

% ---------------------
